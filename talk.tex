\documentclass{beamer}

\usepackage{minted}

\newminted{rust}{bgcolor=bg}
\newminted{c}{bgcolor=bg}

\AtBeginSection[]{
    \begin{frame}
        \begin{center}
            \usebeamerfont{title}\secname
        \end{center}
    \end{frame}
}

\AtBeginSubsection[]{
    \begin{frame}
        \begin{center}
            \usebeamerfont{subtitle}\subsecname
        \end{center}
    \end{frame}
}

\title{FFI in Rust}
\subtitle{Lessons from OpenSSL}
\author[sfackler]{Steven Fackler - sfackler}
\date{December 15, 2016}

\begin{document}
\definecolor{bg}{rgb}{.95,.95,.95}

\frame{\titlepage}

\frame{\tableofcontents}

\section{Basics}

\begin{frame}{What is FFI?}
    \begin{quote}
        ``A foreign function interface (FFI) is a mechanism by which a program written in one programming language can call routines or make use of services written in another.''
    \end{quote}
\end{frame}

\begin{frame}{Why FFI?}
    Lots of code has been written in languages that aren't Rust!

    \begin{itemize}
        \item Use Rust to accelerate portions of programs written in higher level languages. (jni, neon, ruru)
        \item Expose a C interface for your Rust library. (regex)
        \item Interact with libraries exposing C APIs. (openssl, curl, rusqlite)
    \end{itemize}
\end{frame}

\begin{frame}[fragile]{Example}
    \begin{ccode}
struct foo {
    long a;
    char *b;
};
int fizzbuzz(const struct foo *f);
    \end{ccode}
    \vspace{10pt}
    \begin{rustcode}
#[repr(C)]
struct foo {
    a: c_long,
    b: *mut c_char,
}
extern {
    fn fizzbuzz(f: *const foo) -> c_int;
}
    \end{rustcode}
\end{frame}

\end{document}
